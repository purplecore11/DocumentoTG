\fontsize{14}{15}\selectfont
Chile se encuentra situado a lo largo de un margen convergente, en el cual la placa de Nazca se subduce por debajo de la placa Sudamericana, con una convergencia oblicua con orientación N75°E, a una velocidad de $\sim$ 6.7 cm/año \citep*{angermann_space-geodetic_1999,argus_geologically_2011}. Según \cite{bejar-pizarro_asperities_2010} el terremoto de Tocopilla 2007 etc etc etc. La ecuación \eqref{eq:1} es un ejmeplo para realizar una ecuación correctamente.\\
\begin{equation}
	\Delta \Phi_{flat,topo,defo} = \dfrac{4\pi}{\lambda}\cdot\Delta R
	\label{eq:1}
\end{equation}\\
Para matrices realizar la ecuación \eqref{eq:matrix}:\\
\begin{equation}
	\Delta L= \begin{bmatrix}
		sen(\theta)\cdot sen(\alpha) & -sen(\theta)\cdot cos(\alpha) & cos(\theta)
	\end{bmatrix}
	\begin{bmatrix}
		\delta Norte\\
		\delta Este\\
		\delta Up\\
	\end{bmatrix}
	\label{eq:matrix}
\end{equation}\\